การออกแบบโปรแกรมของหุ่นยนต์ฮิวมานอยด์ UTHAI ผู้วิจัยได้ใช้ ROS เป็นเครื่องมือที่ช่วยในการทำงาน
ในวิทยานิพนธ์นี้ส่วนของโปรแกรมจะแบ่งออกเป็น 3 ส่วนหลักคือ
\vspace{-3mm}
\begin{enumerate}[label=\arabic*, leftmargin=1.5cm]
	\setlength\itemsep{-0.25em}
	\item การแสดงผลภาพด้วย RViz
    \item การจำลองการทำงานของหุ่นยนต์
    \item การควบคุมการทำงานของหุ่นยนต์ฮิวมานอยด์อุทัย
\end{enumerate}

ผู้วิจัยได้ใช้ ROS เป็นเครื่องมือที่ช่วยในการทำงานเนื่องจากเป็นสากล และช่วยเพิ่มประสิทธิภาพในการทำงานได้
ในส่วนของการแสดงผลด้วยภาพผ่านโปรแกรม RViz ผู้วิจัยได้เขียนไฟล์ URDF ไว้สำหรับหุ่นยนต์ฮิวมานอยด์ UTHAI
โดยมีได้ใส่ข้อมูลทางพลศาสตร์ของหุ่นยนต์ไว้ด้วย ในส่วนของการจำลองการทำงานของหุ่นยนต์ ผู้วิจัยได้ใช้โปรแกรมจำลองการทำงานเป็น Gazebo
ซึ่งเป็นโปรแกรมที่สามารถใส่ข้อมูลทางฟิสิกส์เข้าไปได้ ช่วยทำให้เห็นภาพก่อนการทำงานจริง
ในส่วนสุดท้ายการควบคุมการทำงานของหุ่นยนต์ฮิวมานอยด์ ผู้จัดทำได้สร้างแพกเกจสำหรับควบคุมการขับเคลื่อนของดิจิตอลเซอร์โวไว้
เพื่อทำให้สะดวกในการใช้งานต่อไป

\subsection*{ข้อเสนอแนะ}
การปรับปรุงควรสร้างไฟล์ให้สามารถรันครั้งเดียวแล้วเปิดทุกโปรแกรมที่ต้องการทำงาน เนื่องจากปัญหาตอนนี้ที่ผู้วิจัยพบเจอคือ
เวลาเปิดโปรแกรมเพื่อที่ทำคำสั่งนั้นต้องใช้หน้าต่างเป็นจำนวนมาก โปรแกรมโดนแบ่งเป็นส่วนย่อยๆหลายๆส่วน จึงควรที่จะรวบรวมให้เป็นโปรแกรมเดียว

