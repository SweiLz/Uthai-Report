การออกแบบโครงสร้างหุ่นยนต์ฮิวมานอยด์ UTHAI มุ่งเน้น 2 ส่วนเป็นหลักคือ
\vspace{-3mm}
\begin{enumerate}[label=\arabic*, leftmargin=1.5cm]
	\setlength\itemsep{-0.25em}
	\item สามารถสร้างขึ้นได้ง่าย
	\item น้ำหนักเบา
\end{enumerate}

ผู้วิจัยจึงเลือกที่จะประยุกต์ใช้เทคนิคการพิมพ์ขึ้นรูปสามมิติด้วยเครื่องพิมพ์สามมิติ ในการขึ้นรูปข้อต่อส่วนต่างๆของหุ่นยนต์ฮิวมานอยด์อุทัย
และก้านต่อได้เลือกใช้วัสดุเป็นคาร์บอนไฟเบอร์ ซึ่งเป็นวัสดุที่มีคุณสมบัติคือ เบา และแข็งแรง เมื่อเทียบกับวัสดุอื่นๆ ทำให้หุ่นยนต์ฮิวมานอยด์อุทัยมีน้ำหนักเบา
การเชื่อมต่อระหว่างข้อต่อ กับก้านต่อคาร์บอนไฟเบอร์ ผู้วัจัยใช้วิธีการบีบเพื่อสร้างแรงเสียดทานในการยึดติด
เนื่องจากการเจาะท่อคาร์บอนไฟเบอร์จะทำให้ใยไฟเบอร์ขาด ซึ่งส่งผลต่อความแข็งแรงของท่อคาร์บอนไฟเบอร์เป็นอย่างมาก
อีกทั้งเมื่อมีการเคลื่อนไหวและรับแรงในแนวต่างๆจะทำให้รูที่เจาะขยายและคลอนได้ส่งผลต่อความแม่นยำโดยรวมของหุ่นยนต์
แต่การยึดติดด้วยวิธีการบีบกับท่อคาร์บอนไฟเบอร์นั้นมีปัญหาเกิดขึ้นคือ มีโอกาสที่จะประกอบโครงสร้างของหุ่นยนต์ไม่ตรงเพราะอาจเกิดการหมุนตามแนวยาวของชิ้นส่วน
และหากใช้งานต่อเนื่องจะทำให้เกิดการหมุนเลื่อนตามแนวยาวของท่อได้

\subsection*{ข้อเสนอแนะ}
การพัฒนาต่อควรปรับปรุงในส่วนนี้การเชื่อมต่อชิ้นส่วนเข้าด้วยกัน อาจจะด้วยวิธีเพิ่มรอยบากเพื่อให้ไม่สามารถหมุนได้
การประกอบโครงสร้างควรใช้วัสดุแผ่นยางบางมาคั่นกลางระหว่างหน้าสัมผัสที่ท่อคาร์บอนไฟเบอร์ซึ่งแผ่นยางจะสัมผัสกับชิ้นส่วนที่พิมพ์จากเครื่องพิมพ์สามมิติ
