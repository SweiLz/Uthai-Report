\subsection{GitHub ของหุ่นยนต์อุทัย}
ไฟล์ข้อมูลทุกอย่างเกี่ยวกับหุ่นยนต์ฮิวมานอยด์อุทัยได้ถูกอัพโหลดขึ้นบนอินเทอร์เนต
โดยอัพโหลดไปไว้ที่ GitHub [https://github.com/UTHAI-Humanoid] และมีการเขียน Wiki การใช้งานเบื้องต้นเอาไว้
สำหรับนักศึกษาหรือนักวิจัยที่ต้องการพัฒนาต่อ สามารถที่จะ Pull request เข้ามาได้
ซึ่งจะช่วยทำให้หุ่นยนต์ฮิวมานอยด์อุทัยสามารถพัฒนาต่อยอดต่อไปได้
\begin{figure}[!ht]
	\centering
	\includegraphics[width=0.40\textwidth]{chapter4/images/uthai_manual/uthai_github.png}
	\caption{GitHub ที่เก็บข้อมูลทั้งหมดของหุ่นยนต์อุทัย}
\end{figure}
\begin{figure}[!ht]
	\centering
	\includegraphics[width=0.60\textwidth]{chapter4/images/uthai_manual/uthai_github2.png}
	\caption{ตัวอย่าง Wiki การใช้งานเบื้องต้นของหุ่นยนต์อุทัย}
\end{figure}


\clearpage
\subsection{ตัวอย่างเฟรมของหุ่นยนต์อุทัย}
เฟรมของหุ่นยนต์อุทัยได้ถูกอัพโหลดให้อยู่บนอินเทอร์เน็ต โดยอยู่ที่ https://github.com/UTHAI-Humanoid/UTHAI-Hardware/tree/master/Mechanics/Frame

\begin{figure}[!ht]
	\centering
	\includegraphics[width=0.85\textwidth]{chapter4/images/uthai_manual/uthai_frame.png}
	\caption{ภาพเฟรมของแต่ละพาร์ทของหุ่นยนต์อุทัย}
\end{figure}
\begin{figure}[!ht]
	\centering
	\includegraphics[width=0.85\textwidth]{chapter4/images/uthai_manual/uthai_frame2.png}
	\caption{ภาพ GitHub ของเฟรมแต่ละพาร์ทของหุ่นยนต์อุทัย}
\end{figure}

\clearpage
\subsection{ตัวอย่างเอกสารข้อมูลของหุ่นยนต์อุทัย}

\begin{figure}[!ht]
    \centering
    \begin{subfigure}[b]{0.45\textwidth}
        \centering
        \includegraphics[width=\textwidth]{chapter4/images/uthai_manual/uthai_assembly.png}
        \caption{หน้าปกคู่มือการประกอบหุ่นยนต์อุทัย}
    \end{subfigure}
    \hfill
    \begin{subfigure}[b]{0.45\textwidth}
        \centering
        \includegraphics[width=\textwidth]{chapter4/images/uthai_manual/uthai_assembly2.png}
        \caption{ตัวอย่างคู่มือการประกอบหุ่นยนต์อุทัย}
    \end{subfigure}
    \caption{UTHAI Assembly Manual}
	\label{fig:uthai_assembly_manual}
\end{figure}
\begin{figure}[!ht]
    \centering
    \begin{subfigure}[b]{0.45\textwidth}
        \centering
        \includegraphics[width=\textwidth]{chapter4/images/uthai_manual/uthai_kinematics.png}
        \caption{หน้าปกรายละเอียดของหุ่นยนต์อุทัย}
    \end{subfigure}
    \hfill
    \begin{subfigure}[b]{0.45\textwidth}
        \centering
        \includegraphics[width=\textwidth]{chapter4/images/uthai_manual/uthai_kinematics2.png}
        \caption{ตัวอย่างรายละเอียดของหุ่นยนต์อุทัย}
    \end{subfigure}
    \caption{UTHAI Kinematics Properties}
	\label{fig:uthai_kinematics_manual}
\end{figure}
\begin{figure}[!ht]
    \centering
    \begin{subfigure}[b]{0.45\textwidth}
        \centering
        \includegraphics[width=\textwidth]{chapter4/images/uthai_manual/uthai_dynamics.png}
        \caption{หน้าปกรายละเอียดของหุ่นยนต์อุทัย}
    \end{subfigure}
    \hfill
    \begin{subfigure}[b]{0.45\textwidth}
        \centering
        \includegraphics[width=\textwidth]{chapter4/images/uthai_manual/uthai_dynamics2.png}
        \caption{ตัวอย่างรายละเอียดของหุ่นยนต์อุทัย}
    \end{subfigure}
    \caption{UTHAI Dynamics Properties}
	\label{fig:uthai_dynamics_manual}
\end{figure}

\clearpage
\subsection{ภาพรวมระบบพื้นฐานของหุ่นยนต์อุทัย}
ระบบพื้นฐานที่ผู้วิจัยได้ทำการออกแบบก็เพื่อที่จะทำให้คนที่เข้ามาพัฒนาต่อยอดได้อย่างสะดวก และเป็นระบบระเบียบ
มีรูปแบบแบบแผน ซึ่งผู้วิจัยได้วางระบบนี้ขึ้นมาจากประสบการณ์การทำงานของผู้วิจัยเอง รวมถึงได้ค้นคว้าหาความรู้เพิ่มเติม
จนคิดว่าระบบนี้จะทำให้หุ่นยนต์ฮิวมานอยด์สามารถพัฒนาต่อได้ง่าย และเป็นประโยชน์ต่อผู้วิจัยท่านอื่นที่ต้องการนำระบบพื้นฐานนี้ไปใช้งาน

ระบบพื้นฐานที่ผู้วิจัยออกแบบขึ้นมานั้น จะประกอบไปด้วยส่วนสำคัญอยู่ทั้งหมด 7 ส่วน คือ
\vspace{-10pt}
\begin{enumerate}[label=\arabic*., leftmargin=2.5cm]
    \setlength\itemsep{-0.25em}
    \item UTHAI-Documents
    \item UTHAI-Hardware
    \item UTHAI-Common
    \item UTHAI-MPPC
    \item UTHAI-OPC
    \item UTHAI-msgs
    \item UTHAI-Tools
\end{enumerate}

\clearpage
\subsubsection*{UTHAI-Documents}
ในส่วนนี้คือส่วนของงานเอกสารต่างๆที่เกี่ยวข้องกับหุ่นยนต์ฮิวมานอยด์

\paragraph*{Reports-Hardware}
ใช้สำหรับเก็บรายงานวิทยานิพนธ์ทั้งฉบับร่างและฉบับสมบูรณ์ที่เกี่ยวข้องกับ การปรับเปลี่ยนโครงสร้างทางกล และทางไฟฟ้าของหุ่นยนต์ฮิวมานอยด์อุทัย
\subparagraph*{- การออกแบบโครงสร้างและพัฒนาระบบพื้นฐานสำหรับหุ่นยนต์ฮิวมานอยด์เพื่อการศึกษาและวิจัย}

\paragraph*{Reports-Software}
ใช้สำหรับเก็บรายงานวิทยานิพนธ์ทั้งฉบับร่างและฉบับสมบูรณ์ที่เกี่ยวข้องกับการเพิ่มความสามารถของหุ่นยนต์ฮิวมานอยด์อุทัย
\subparagraph*{- การพัฒนาระบบการเคลื่อนที่สำหรับหุ่นยนต์ฮิวมานอยด์}

\paragraph*{Wiki}
ใช้สำหรับเก็บ Tutorial ที่เกี่ยวกับการใช้งานของหุ่นยนต์ฮิวมานอยด์อุทัย

\begin{itemize}\setlength\itemsep{-0.3em}
	\setlist{leftmargin=5.5cm}
	\item ศึกษาเกี่ยวกับส่วนประกอบของหุ่นยนต์ฮิวมานอยด์
	\item ศึกษาทฤษฏีที่เกี่ยวข้องกับของมนุษย์
	\item ศึกษาทฤษฏีที่เกี่ยวข้องกับหุ่นยนต์ฮิวมานอยด์
	\item ศึกษาความแตกต่างระหว่างมนุษย์กับหุ่นยนต์ฮิวมานอยด์
	\item ศึกษาวิธีการและวัสดุที่ใช้ในการสร้างหุ่นยนต์
	\item ศึกษาระบบที่ใช้ช่วยในการพัฒนาหุ่นยนต์
	\item ศึกษาระบบที่ใช้ในการจำลองการทำงานของหุ่นยนต์
	\item ศึกษาการใช้งาน ROS พื้นฐาน
\end{itemize}



\subsubsection*{UTHAI-Hardware}
ที่เก็บรายละเอียดเกี่ยวกับโครงสร้างทั้งทางกลและทางไฟฟ้า วิธีการประกอบ รายระเอียดการต่อสายไฟ

\subsubsection*{UTHAI-Common}
เก็บเกี่ยวกับ ROS Package ที่เป็น model ของตัวหุ่นยนต์พร้อม Simulation

\subsubsection*{UTHAI-MPPC}
เครื่อง Motion and Perception PC
เก็บเกี่ยวกับ ROS Package ที่ใช้ใน Odroid ใช้สั่งการ ทำสมดุล

\subsubsection*{UTHAI-OPC}
เครื่อง Operating PC
เก็บเกี่ยวกับ ROS Package ที่ใช้สำหรับเข้าไปดูการทำงานและควบคุมหุ่นยนต์ ดูสถานะต่างๆ Moveit Gazebo

\subsubsection*{UTHAI-msgs}
เป็นที่เก็บ message ที่สร้างขึ้นมาเอง รวมไปถึง service และ action ด้วย

\subsubsection*{UTHAI-Tools}
เก็บเครื่องมือที่ใช้ช่วยในการพัฒนาหุ่นยนต์

