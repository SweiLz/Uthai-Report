% ************************** Thesis Chapter1 **********************************
\chapter{บทนำ}
%%%%%%%%%%%%%%%%%%%%%%%%%%%%%%%%%%%%%%%%%%%%%%%%%%%%%%%%%%%%%%%%%%%%%%%%%%%%%%%
\section{ที่มาและความสำคัญ}
หุ่นยนต์ฮิวมานอยด์เป็นหุ่นยนต์ที่สร้างขึ้นเพื่อเลียนแบบสรีระร่างกายของมนุษย์ 
ซึ่งมีข้อจำนวนมากเพื่อให้มีการเคลื่อนไหวคล้ายมนุษย์ ลักษณะเด่นของหุ่นยนต์ฮิวมานอยด์คือ
การเคลื่อนที่ด้วยขาสองข้างด้วยการเคลื่อนที่โดยการใช้ขานั้น ทำให้หุ่นยนต์ฮิวมานอยด์สามารถเคลื่อนที่ได้
อย่างคล่องแคล่วในทุกสภาพพื้นผิว ทั้งทางเรียบ ทางขรุขระหรือพื้นต่างระดับ
\footnote{การเคลื่อนที่ของหุ่นยนต์รูปแบบต่างๆ ราชภัฏสวนดุสิต}
ซึ่งนั่นทำให้หุ่นยนต์ที่เดินสองขาแตกต่างจากหุ่นยนต์ที่เคลื่อนที่ด้วยล้อ ด้วยโครงสร้างของหุ่นยนต์ที่คล้ายมนุษย์นั้นเอง 
จึงทำให้หุ่นยนต์ฮิวมานอยด์สามารถทำงานได้หลากหลายและยืดหยุ่น สามารถที่จะใช้อุปกรณ์ทั่วไปที่ถูกออกแบบขึ้นมาเพื่อใช้กับมนุษย์ได้
ซึ่งหมายความว่าในอนาคตนั้นหุ่นยนต์ฮิวมานอยด์สามารถที่จะทำงานทดแทนแรงงานของมนุษย์ได้
\footnote{ณัฐพงษ์ วารีประเสริฐ และณรงศ์ ล่ำดี (2552: 374)}
งานที่หุ่นยนต์ฮิวมานอยด์จะเข้ามาทดแทนแรงงานของมนุษย์นั้น จะเป็นงานที่ต้องทำซ้ำๆ จนเกินความเมื่อยล้า
งานที่อยู่ในพื้นที่อันตรายหรือที่เสี่ยงต่อการเกิดอุบัติเหตุ

สถาบันวิจัยหลายแห่งทั่วโลกกำลังให้ความสนับสนุนด้านการศึกษาวิจัยและพัฒนาหุ่นยนต์ฮิวมานอยด์
เพื่อให้ทำภารกิจต่างๆ ยกตัวอย่างเช่น DARPA Robotics Challenge (DRC)
\footnote{DARPA 2015 [https://www.darpa.mil/about-us/about-darpa]}
เป็นรายการแข่งขันหุ่นยนต์กึ่งอัตโนมัติเพื่อทำภารกิจกู้ภัยในสถานการณ์ภัยพิบัติที่อันตราย
ซึ่งสถาบันวิจัยหุ่นยนต์ทั่วโลกได้ส่งหุ่นยนต์ฮิวมานอยด์ของตนเข้าร่วมการแข่งขัน ในปัจจุบันได้มีการพัฒนาหุ่นยนต์ฮิวมานอยด์ขึ้นมาหลากหลายตัวเช่น
ASIMO, HRP-3, LOLA และ WATHLETE-1 การพัฒนาหุ่นยนต์ฮิวมานอยด์นั้นได้ก่อให้เกิดงานศึกษาวิจัย
และทฏษฏี ต่อยอด ต่างๆมากมาย เช่น
การวางแผนการเดิน การเดินแบบสถิต การเดินแบบพลวัต การติดต่อสื่อสารของระบบ การมองเห็นและการประมวลผลภาพ
การพูดคุยโต้ตอบกับมนุษย์ ปัญญาประดิษฐิ์ ฯลฯ ซึ่งงานวิจัยเหล่านี้สามารถที่จะนำไปประยุกต์ใช้กับระบบหุ่นยนต์ระบบอื่นๆได้ 
แม้ว่าจะมีการพัฒนาหุ่นยนต์ฮิวมานอยด์มามากมายแล้ว แต่การเริ่มต้นทำงานวิจัยที่มีความเกี่ยวข้องกับหุ่นยนต์ฮิวมานอยด์นั้น
ต้องใช้ความรู้ความสามารถ เครื่องมือ ระยะเวลา งบประมาณ และ
ความพยายามเป็นอย่างมาก การสร้างหุ่นยนต์ฮิวมานอยด์ขึ้นมาใหม่นั้นต้องใช้งบประมาณสูง ดั้งนั้นการสร้างระบบจำลองของหุ่นยนต์ฮิวมานอยด์ขึ้นมาเป็นระบบพื้นฐาน 
ให้มีความพร้อมสำหรับการพัฒนาต่อยอดแก่นักศึกษาหรือนักวิจัย จะช่วยประหยัดเวลาและงบประมาณที่ต้องใช้ได้อย่างมาก
ซึ่งนั่นหมายความว่านักวิจัยจะสามารถทำงานวิจัยได้อย่างมีประสิทธิภาพมากขึ้น

ในงานวิจัยนี้เป็นการออกแบบหุ่นยนต์ฮิวมานอยด์และพัฒนาระบบพื้นฐานของหุ่นยนต์ฮิวมานอยด์
สำหรับให้นักศึกษาหรือนักวิจัยสามารถพัฒนาต่อยอดได้ โดยหุ่นยนต์ฮิวมานอยด์ที่ออกแบบมานั้น 
สามารถที่จะปรับปรุง แก้ไข ดัดแปลงได้ง่าย ตัวโครงสร้างจะใช้เป็น พลาสติก PLA ที่สามารถขึ้นรูปได้
โดยการใช้เครื่องพิมพ์สามมิติ มีเซนเซอร์ตรวจการสัมผัสพื้นที่ฝ่าเท้าของหุ่นยนต์ มีเซนเซอร์สำหรับการวัดมุมเอียง
ที่ลำตัวของหุ่นยนต์ และเพื่อที่จะทำให้ง่ายต่อการศึกษาทำความเข้าใจ บำรุงรักษา 
จึงได้มีการจัดทำคู่มือและเอกสารวิธีการใช้งานอย่างชัดเจน โดยจะเก็บในรูปแบบของเอกสารออนไลน์


\clearpage
%%%%%%%%%%%%%%%%%%%%%%%%%%%%%%%%%%%%%%%%%%%%%%%%%%%%%%%%%%%%%%%%%%%%%%%%%%%%%%%
\section{วัตถุประสงค์}
ผู้วิจัยมีวัตถุประสงค์ในการทำวิทยานิพนธ์เกี่ยวกับหุ่นยนต์ฮิวมานอยด์แพลตฟอร์มนี้ขึ้นมาก็เพื่อที่จะ
ออกแบบโครงสร้างของหุ่นยนต์ฮิวมานอยด์ที่สามารถแก้ไขปรับเปลี่ยนได้ง่าย พัฒนาระบบพื้นฐาน
ระบบจำลองสำหรับหุ่นยนต์ฮิวมานอยด์เพื่อใช้ในการศึกษาวิจัย รวบรวมเครื่องมือที่เป็นประโยชน์ต่อการพัฒนาหุ่นยนต์
และจัดทำเอกสารออนไลน์ ให้บุคคลที่สนใจสามารถเข้ามาศึกษาได้

%%%%%%%%%%%%%%%%%%%%%%%%%%%%%%%%%%%%%%%%%%%%%%%%%%%%%%%%%%%%%%%%%%%%%%%%%%%%%%%
\section{ประโยชน์ที่คาดว่าจะได้รับ}
\begin{enumerate}[label=\thesection.\arabic*, leftmargin=1.5cm]
	\setlength\itemsep{-0.25em}
	\item มีต้นแบบหุ่นยนต์ฮิวมานอยด์สำหรับใช้ในงานวิจัยแขนงต่างๆ
	\item มีระบบพื้นฐานสำหรับพัฒนาหุ่นยนต์ฮิวมานอยด์รุ่นใหม่ในสถาบัน
	\item มีระบบจำลองสำหรับจำลองการทำงานของหุ่นยนต์ฮิวมานอยด์
	\item มีแหล่งรวบรวมเครื่องมือสำหรับการพัฒนาหุ่นยนต์
	\item มีคู่มือ เอกสาร วิธีการใช้งาน และรายละเอียดของหุ่นยนต์สำหรับพัฒนาต่อยอด
\end{enumerate}

%%%%%%%%%%%%%%%%%%%%%%%%%%%%%%%%%%%%%%%%%%%%%%%%%%%%%%%%%%%%%%%%%%%%%%%%%%%%%%%
\section{ขอบเขตการดำเนินงาน}
\begin{enumerate}[label=\thesection.\arabic*, leftmargin=1.5cm]
	\setlength\itemsep{-0.25em}
	\item ใช้ ROS เป็นกรอบการทำงานสำหรับพัฒนาระบบพื้นฐาน
	\item ออกแบบโครงสร้างให้มีความแข็งแรง สามารถรองรับน้ำหนักอุปกรณ์ที่ติดตั้งอยู่บนตัวหุ่นยนต์ได้
	\item น้ำหนักของหุ่นยนต์รวมกันทั้งตัว ไม่เกิน 5 กิโลกรัม
	\item ใช้ Solidworks 3D เป็นโปรแกรมสำหรับออกแบบโครงสร้าง และคำนวณ
	\item หุ่นยนต์มีความสูงไม่ต่ำกว่า 100 เซนติเมตร และสูงไม่เกิน 120 เซนติเมตร
	\item หุ่นยนต์มี 2 แขน 2 ขา มีองศาอิสระของขาข้างละ 6 และแขนข้างละ 2 องศาอิสระ
	\item หุ่นยนต์สามารถทำงานได้ภายในสภาพแวดล้อมแบบปิด
	\item หุ่นยนต์ใช้พลังงานจากแหล่งจ่ายพลังงานที่มีขนาดแรงดันไฟฟ้า 12 โวลต์
	\item หุ่นยนต์ใช้ตัวขับเคลื่อนแบบดิจิตอลสำหรับแต่ละข้อต่อเป็น Dynamixel Digital Servo
	\item ใช้ Gazebo สำหรับจำลองระบบของหุ่นยนต์
	\item ติดตั้งเซนเซอร์วัดการกด (Ground contact) ที่ฝ่าเท้าของหุ่นยนต์
	\item ติดตั้งเซนเซอร์วัดมุมเอียง (IMU) ที่บริเวณลำตัวของหุ่นยนต์
	\item จัดทำคู่มือ เอกสารการใช้งาน และรายละเอียดส่วนประกอบของหุ่นยนต์
\end{enumerate}

\clearpage
%%%%%%%%%%%%%%%%%%%%%%%%%%%%%%%%%%%%%%%%%%%%%%%%%%%%%%%%%%%%%%%%%%%%%%%%%%%%%%%
\section{ภาพรวมของระบบและขั้นตอนการดำเนินงาน}
งานวิจัยนี้การดำเนินงานวิจัยถูกแบ่งออกเป็นสามส่วน คือ ส่วนที่หนึงส่วนโครงสร้างของหุ่นยนต์ฮิวมานอยด์ เป็นส่วนที่ทำในส่วนของการขึ้นรูปชิ้นงาน
ออกแบบโมเดลสามมิติ รวมไปถึงระบบอิเล็กทรอนิกส์ ติดตั้งบอร์ดและเซนเซอร์ไว้ตามจุดต่างๆ เพื่อสร้างโครงสร้างของหุ่นยนต์ให้สามารถรองรับการเดินได้
ส่วนที่สองส่วนโปรแกรมของหุ่นยนต์ฮิวมานอยด์ เป็นส่วนที่ทำในส่วนของการสั่งการตัวขับเคลื่อนต่างๆ อ่านค่าสถานะเซนเซอร์จากคอนโทลเลอร์
รวมไปถึงระบบจำลองการทำงานของหุ่นยนต์ และส่วนที่สามส่วนการออกแบบระบบพื้นฐานเพื่อการพัฒนาต่อยอด ส่วนนี้จะเป็นส่วนที่ทำให้ผู้ที่จะมาวิจัยต่อยอดสามารถทำงานได้ง่ายขึ้น
จัดการเอกสารคู่มือการใช้งานต่างๆให้เป็นระบบระเบียบ สามารถแยกขั้นตอนการทำงานของแต่ละส่วนออกเป็นข้อดังนี้

\subsection*{ศึกษาค้นคว้าเอกสารและงานวิจัยที่เกี่ยวข้อง}
\begin{itemize}\setlength\itemsep{-0.3em}
	\setlist{leftmargin=1.5cm}
	\item ศึกษาเกี่ยวกับส่วนประกอบของหุ่นยนต์ฮิวมานอยด์
	\item ศึกษาทฤษฏีที่เกี่ยวข้องกับของมนุษย์
	\item ศึกษาทฤษฏีที่เกี่ยวข้องกับหุ่นยนต์ฮิวมานอยด์
	\item ศึกษาความแตกต่างระหว่างมนุษย์กับหุ่นยนต์ฮิวมานอยด์
	\item ศึกษาวิธีการและวัสดุที่ใช้ในการสร้างหุ่นยนต์
	\item ศึกษาระบบที่ใช้ช่วยในการพัฒนาหุ่นยนต์
	\item ศึกษาระบบที่ใช้ในการจำลองการทำงานของหุ่นยนต์
	\item ศึกษาการใช้งาน ROS พื้นฐาน
\end{itemize}
\subsection*{1) ส่วนโครงสร้างของหุ่นยนต์ฮิวมานอยด์}
\begin{itemize}\setlength\itemsep{-0.3em}
	\setlist{leftmargin=1.5cm}
	\item ออกแบบโครงสร้างของหุ่นยนต์ฮิวมานอยด์
	\item จัดสร้างโครงสร้างหุ่นยนต์ฮิวมานอยด์
	\item ทดสอบการทำงานของหุ่นยนต์ฮิวมานอยด์
\end{itemize}
\subsection*{2) ส่วนโปรแกรมของหุ่นยนต์ฮิวมานอยด์}
\begin{itemize}\setlength\itemsep{-0.3em}
	\setlist{leftmargin=1.5cm}
	\item ออกแบบโปรแกรมของหุ่นยนต์ฮิวมานอยด์
	\item สร้างโปรแกรมส่วนล่างสำหรับหุ่นยนต์ฮิวมานอยด์
	\item สร้างโปรแกรมส่วนบนสำหรับหุ่นยนต์ฮิวมานอยด์
	\item ทดสอบการทำงานของโปรแกรม
\end{itemize}
\subsection*{3) ส่วนการออกแบบระบบพื้นฐานเพื่อการพัฒนาต่อยอด}
\begin{itemize}\setlength\itemsep{-0.3em}
	\setlist{leftmargin=1.5cm}
	\item ติดตั้งระบบ
	\item วางระบบพื้นฐาน
	\item รวบรวมเครื่องมือที่เป็นประโยชน์
	\item จัดทำคู่มือ
\end{itemize}

%%%%%%%%%%%%%%%%%%%%%%%%%%%%%%%%%%%%%%%%%%%%%%%%%%%%%%%%%%%%%%%%%%%%%%%%%%%%%%%
